% Introduction
% ============

\chapter{Introduction}
\label{ch:introduction}

\section{Qu'est-ce que la calculabilité}
\label{sec:qu_est-ce_la_calculabilite}

\paragraph{}
La calculabilité c'est l'étude des limites de l'informatique. Il faut bien
faire attention à faire la différence entre les limites théoriques et les limites
pratiques. Pour la calculabilité, on s'occupe des limites théoriques alors que pour 
la complexité on s'occupe des limites pratiques. La complexité
détermine la frontière entre ce qui est faisable et infaisable en pratique.
La question principale de la calculabilité est: ``quels sont les problèmes qui peuvent
être résolus par un programme?''. La caractéristique de calculabilité ne donne aucune 
autre information sur le programme que de la preuve de son existence.

\paragraph{} Le but est donc de tracer des frontières entre les programmes calculables,
non calculables et non calculables en pratique.

\paragraph{}
Cela nous permet de savoir quand ça ne sert à rien d'essayer de résoudre un problème.
De plus, on est conscient de la complexité intrinsèque d'un problème.
% paragraph  (end)

\subsection{Exemples de limites}
\label{subsec:exemples_limites}

De nombreuses limites existent en informatique, par exemple il est impossible de déterminer quand un programme se termine. Ou bien on ne peut pas déterminer qu'un programme est écrit sans bugs.

Mais des limites existent dans bien d'autres domaines que l'informatique. 
Par exemple dans plusieurs champs de la physique, on retrouve les lois de la 
thermodynamique dans ce rôle. En effet on ne peut créer de l'énergie à partir de rien.

En biologie, les lois de Mendel forment une limite. Un individu ne peut recevoir 
d'autres gènes que ceux de sa propre famille. %Verifier si c'est bien pertinent

\section{Notion de problème}
\label{sec:notion_de_probl_me}

\paragraph{}
Premièrement, on doit parler de la notion de problème.
Attention, il ne faut pas confondre un problème avec un programme.
Les caractéristiques d'un problème sont:

\begin{itemize}
	\item un problème est générique : il s'applique à un ensemble de données.
	\item pour chaque donnée particulière, il existe une réponse.
\end{itemize}
On représente un problème dans le cours par une fonction. Donc dans le cours,
la description d'un problème est équivalente à la description d'une fonction.
% paragraph  (end)
% subsection notion_de_probl_me (end)

\section{Notion de programme}
\label{sec:notion_de_programme}

Un programme est une ``procédure effective'', c'est-à-dire exécutable par une machine.
Il existe plein de formalismes permettant la description de ``procédure effective''.

% subsection notion_de_programme (end)

\subsection{Résultats principaux}
\label{subsec:r_sultat_principaux}

\begin{itemize}
	\item Équivalence des langages de programmation (complets).
	\item Problème non calculable : il existe des problèmes qui ne peuvent
		être résolus par un programme. Ex:
        détection de virus,
        équivalence de programme,
        déterminer si un polynôme à coefficients entiers à des racines entières, ...
	\item Problème intrinsèquement complexe. (Voir complexité) Les problèmes
		qui ont une complexité supérieure ou égale à l'exponentielle. Dans
		ce cas, même l'amélioration des ordinateurs n'influence presque pas
		la taille de l'entrée possible.
\end{itemize}

% subsection r_sultat_principaux (end)

\section{Détection de virus}
\label{sec:d_tection_de_virus}
On veut déterminer si un programme P avec une entrée D est nuisible.

Spécification du programme detecteur(P,D):\\
\textbf{Préconditions :} un programme P et une donnée D\\
\textbf{Postconditions :} ``Mauvais'' si P(D) est nuisible,
		``Bon'' sinon.
Il faut aussi que detecteur ne soit pas nuisible.

On va créer un programme drole(P) et essayer de détecter s’il est nuisible.

\begin{lstlisting}
drole(P)
if detecteur(P,P) = ``Mauvais''
	then stop
else infecter un autre programme en y inserant P
\end{lstlisting}

Testons \lstinline|drole(drole)|.
\begin{lstlisting}
drole(drole)
if detecteur(drole, drole) = ``Mauvais''
	then stop
else infecter un autre programme en y inserant drole
\end{lstlisting}

\begin{itemize}
	\item Supposons que \lstinline|drole(drole)| soit nuisible.
      Lorsqu'on exécute, \lstinline|drole(drole)|
      \lstinline|detecteur(drole,drole)| n'infecte rien car \lstinline|detecteur| n'est pas nuisible.
      Comme \lstinline|detecteur| retourne ``Mauvais'',
      le programme s'arrête.
      Rien a donc été infecté, ce qui est contradictoire avec le fait que \lstinline|drole(drole)| est nuisible.
	\item Si par contre il n'est pas nuisible alors \lstinline|detecteur(drole,drole)|
      ne va pas retourner Mauvais et on va arriver dans le \lstinline|else|.
      On a donc infecter un autre programe que qui contredit le fait que \lstinline|drole(drole)| n'est pas nuisible.
\end{itemize}

On a donc une contradiction dans tous les cas ce qui implique que le programme \lstinline|drole| ne peut
exister, ce qui implique que le programme detecteur non plus.
% paragraph  (end)
% subsection d_tection_de_virus (end)

% section introduction (end)Introduction


