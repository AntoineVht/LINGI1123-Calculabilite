% Concepts
% =========

\chapter{Concepts}
\label{sec:concepts}

% Dans cette partie il y a pas moyen de synthétiser beaucoup.

\section{Ensembles, langages, relations et fonctions}
\label{sub:ensembles_langages_relations_et_fonctions}

\subsection{Ensembles}
\label{ssub:ensembles}
Un ensemble est une collection d'objets, sans répétition, appelés les éléments
de l'ensemble.\\

Notation :
\begin{itemize}
	\item Ensemble fini : $\{ 0, 1, 2\}$
	\item Ensemble infini : $\{ 0, 1, 2, \ldots\}$
	\item Produit cartésien : $A \times B$
    \item Le nombre d'éléments : $|A|$
    \item Ensemble des sous-ensembles : $2^A$ ou $\mathcal{P}(A)$, e.g. $2^{\{2,4\}} = \{\{\}, \{2\}, \{4\}, \{2,4\}\}$.
      On peut remarquer que $|2^A| = 2^{|A|}$
	\item Complément : $\stcomp{A}$
\end{itemize}

% subsubsection ensembles (end)

\subsection{Langages}
\label{ssub:Langages}
Notation :
\begin{itemize}
    \item une \emph{chaîne de caractère} ou un \emph{mot} : séquence FINIE de symboles.
		abceced, 010101101
	\item chaîne de caractères vide : $\epsilon$
    \item un \emph{alphabet} $\Sigma$ est un ensemble de symboles. $\Sigma = \{1, 2\}$
    \item un \emph{langage} est un ensemble de mots constitués de symboles d'un alphabet
		donné.
	\item ensemble de tous les mots possibles sur l'alphabet $\Sigma$ : $\Sigma ^*$
\end{itemize}

% subsubsection Langages (end)

\subsection{Relations}
\label{ssub:relations}
Soient $A$, $B$ des ensembles.
\begin{itemize}
	\item Une \emph{relation} $R$ sur $A$, $B$ est un sous-ensemble de $A \times B$. C'est-à-dire
		un ensemble de paires $\langle a,b \rangle$ avec $a\in A$, $b\in B$.
	\item On peut définir une relation par sa table
	\item On peut écrire $\langle a,b \rangle \in R$ ou $aRb$ ou $R(a,b)$.
\end{itemize}

% subsubsection relations (end)

\subsection{Fonctions}
\label{ssub:fonctions}
Soient $A$, $B$ des ensembles.
\begin{itemize}
  	\item Une \emph{fonction} $f \colon A \rightarrow B$ est une relation telle que pour tout $a \in
	A$, il existe au plus un $b \in B$ tel que $\langle a,b \rangle \in f$
	\item écrire $f(a)=b$ est équivalent à $\langle a,b \rangle \in f$
	\item Si il n'existe pas de $b \in B$ tel que $f(a)=b$ alors $f(a)$ est indéfini,
		$f(a) = \perp$
\end{itemize}

Soit $f\colon A \to B$, on définit le \emph{domain} et l'\emph{image} respectivement comme suit
\begin{align*}
  \dom(f)   & = \{\, a \in A \mid f(a) \neq \bot \,\},\\
  \image(f) & = \{\, b \in B \mid \exists a \in A : b = f(a) \,\}.
\end{align*}

Si $\dom(f) \subseteq A$, $f$ est appelée \emph{partielle} et si $\dom(f) = A$, $f$ est appelée \emph{totale}.
Notez qu'avec cette définition, une fonction totale est partielle.
Pour dire que $\dom(f) \subset A$, c'est plus explicite de dire que $f$ n'est pas totale que dire que $f$ est partielle.

Une fonction est \emph{surjective} si $\image(f) = B$ et \emph{injective} si $\forall a,a' \in A$, $a \neq a' \Rightarrow f(a) \neq f(a')$.

Une fonction est \emph{bijective} si elle est totale, injective et surjective.

On utilise aussi le concept d'\emph{extension} :
$f: A \rightarrow B$ est une extension de $g: A \rightarrow B$ si $\forall x \in A : g(x)\neq \perp \Rightarrow f(x) = g(x)$.
Autrement dit, $f$ a la même valeur que $g$ partout où $g$ est définie.

\paragraph{Définition d'une fonction}
\label{par:d_finition_d_une_fonciton}
On définit une fonction par sa table qui peut-être infinie.\\
On peut définir la table de plusieurs façons :
\begin{itemize}
	\item Par un texte fini déterminant sans contradiction ni ambigüité le contenu
		de la table.
	\item Par un algorithme ex : $f(x) = 2x^3+5$
	\item Écrire toutes les paires de la relation.
\end{itemize}
Attention, il n'est pas nécessaire de décrire ou de connaître un moyen de la calculer
pour pouvoir la définir. Ex : $f(x) = 1$ s'il y a de la vie autre part que sur terre,
$0$ sinon.
% paragraph d_finition_d_une_fonction (end)
% subsubsection fonctions (end)
% subsection ensembles_langages_relations_et_fonctions (end)

\section{Ensemble énumérable}
\label{sub:ensemble_num_rables}

Avant de dire ce qu'est un ensemble énumérable, on doit savoir que deux ensembles
ont le même cardinal s’il existe une bijection entre eux.

\begin{mydef}[Ensemble énumérable]
	Un ensemble est énumérable ou dénombrable s'il est fini ou s'il a le même cardinal que $\mathbb{\N}$. \\
	Quelques propriétés :
\end{mydef}

\begin{myprop}
	Tout sous-ensemble d'un ensemble énumérable est énumérable.
\end{myprop}

\begin{myprop}
	L'union et l'intersection de deux ensembles énumérables sont énumérables.
\end{myprop}

\begin{myprop}
	L'union d'une infinité énumérable d'ensembles énumérables est énumérable.
    \begin{proof}
      La démonstration est similaire à la démonstration de l'énumérabilité de $\mathbb{Q}$.
      On met chaque ensemble en ligne, il y a un nombre énumérable de lignes qu'on peut numéroter en parcourant le tableau en zigzag
      en partant de $(0,0)$.
    \end{proof}
\end{myprop}
L'union d'une infinité non-énumérable d'ensembles énumérable peut ne pas être énumérable.
Par exemple, l'union des singletons $\{x\}$ pour tout réel $x$ forme l'ensemble des réels $\R$ qui n'est pas énumérable:
\[ \bigcup_{x \in \R} \{x\} = \R. \]

\begin{myrem}
  Une bonne intuition à avoir:
  Tout ensemble dont les éléments peuvent être représentés de manière finie est énumérable.

  Par exemple, les rationnels, même s'ils ont une représentation décimale infinie peuvent être représentés de manière finie en fraction d'entiers.
\end{myrem}

Quelques ensembles non énumérables :
\begin{myexem}
 L'ensemble $\R$
 \begin{proof}
   Voir ci-dessous.
 \end{proof}
\end{myexem}

\begin{myexem}
 L'ensemble des sous-ensembles de $\N$, $\mathcal{P}(\N)$
 \begin{proof}
   Rappelons nous tout d'abord que comme $\N$ est infini, ses sous-ensemble peuvent l'être aussi.
   Il est adéquat de visualiser un ensemble à l'aide d'un mot binaire
   où le bit $i$ vaut 1 si le $i$ième élément est pris dans le sous-ensemble.
   Par exemple, le sous-ensemble $\{1,4\}$ de $\{1,3,4\}$ peut être représenté par $101$.
   Seulement, comme il y a un nombre infini de nombre entiers, on a mot binaire infini.
   Par exemple, les nombres pairs, c'est le mot $10101010101\ldots$:
   \begin{verbatim}
0123456789...
1010101010... nombres pairs

0123456789...
0011010100... nombres premiers

0123456789...
0001001001... multiples de 3 non-nuls
    \end{verbatim}
   On voit maintenant la bijection entre les sous-ensembles de $\N$ et $[0,1]$.
   En effet, on peut associer par exemple l'ensembe des nombres pairs au réel $0.101010101\ldots$.
   Plus précisément on associe le sous-ensemble de $\N$ représenté par le mot $m$, le réel entre $0$ et $1$ dont l'écriture binaire est $0.m$.
 \end{proof}
\end{myexem}

\begin{myexem}
 L'ensemble des chaînes infinies de caractère sur un alphabet fini
 \begin{proof}
   On utilise le même raisonnement que pour les sous-ensembles de $\N$ sauf que pour un alphabet de $k$ symboles,
   on utilise la représentation des réels en base $k$.
 \end{proof}
\end{myexem}

\begin{myexem}
  \label{exem:fNN}
 L'ensemble des fonctions de $\N$ dans $\N$ (Cas important)
 \begin{proof}
   Si c'était dénombrable, soit $f_0, f_1, f_2, \ldots$ leur dénombrement.
   On construit le tableau
   \[
     \begin{array}{ccccc}
       f_0(0) & f_0(1) & f_0(2) & f_0(3) & \cdots\\
       f_1(0) & f_1(1) & f_1(2) & f_1(3) & \cdots\\
       f_2(0) & f_2(1) & f_2(2) & f_2(3) & \cdots\\
       \vdots & \ddots & \ddots & \ddots & \ddots\\
     \end{array}
   \]
   et on conclut par diagonalisation de façon semblable à $\R$
   en construisant $d\colon \N \to \N$ tel que $d(k) = f_k(k)$.
 \end{proof}
\end{myexem}

% subsection ensemble_num_rables (end)

\section{Cantor}
\label{sub:cantor}
On va montrer qu'il existe des ensembles non énumérables par diagonalisation. Ex $\R$.
\begin{myexem}
	Exemple de démonstration par diagonalisation:
	\begin{enumerate}
		\item Construire une table : Liste de tous les grands
			mathématiciens \\
			\begin{tabular}{l}
				\textit{\textbf{D}}E MORGAN \\
				A\textit{\textbf{B}}EL\\
				BO\textit{\textbf{O}}LE\\
				BRO\textit{\textbf{U}}WER\\
				SIER\textit{\textbf{P}}INSKI\\
				WEIER\textit{\textbf{S}}TRASS\\
			\end{tabular}
		\item Sélectionner la diagonale : $diag = $DBOUPS
		\item Modifier l'élément égal à la diagonale : $diag' =$ CANTOR
		\item Montrer que l'élément n'est pas dans la liste $\Rightarrow$ Contradiction
		\item Conclusion :
			\begin{itemize}
				\item Soit on sait que la liste est complète\\
					$ \Rightarrow$ CANTOR n'est pas un grand
				mathématicien (cas utilisé pour démontrer
				halt).
				\item Soit on sait que CANTOR est un grand
					mathématicien \\
					$ \Rightarrow$ la liste est incomplète
				(cas utilisé pour la diagonalisation de CANTOR)
			\end{itemize}
	\end{enumerate}
\end{myexem}

\begin{mytheo}[Diagonalisation de Cantor]
	Soit $E = \{ x \text{ réel }| 0<x\leq1\}$, $E$ est non énumérable.
\end{mytheo}

\paragraph{Démonstration :}
On va montrer qu'un nombre $d'$ n'est pas dans l'énumération alors qu'on sait
que $d'$ est un nombre réel compris entre 0 et 1.

On suppose $E$ énumérable. Donc il existe une énumération des éléments de $E$,
$x_0, x_1,\dots,X_k,\dots$. On peut représenter un nombre $x_k$ comme étant une
suite de chiffre $x_{ki}$ : $x_k = 0.x_{k0}x_{k1}\dots x_{kk}\dots$.

\begin{enumerate}
	\item On peut donc construire une table infinie : \\
		\begin{tabular}{|c||c|c|c|c|c|c|}
			\hline
			& 1 digit & 2 digit & 3 digit & ... & k+1 digit & ... \\
			\hline
			$x_0$ & $x_{00}$ & $x_{01}$ & $x_{02}$ & ... & $x_{0k}$ & ... \\
			$x_1$ & $x_{10}$ & $x_{11}$ & $x_{12}$ & ... & $x_{1k}$ & ... \\
			$x_2$ & $x_{20}$ & $x_{21}$ & $x_{22}$ & ... & $x_{2k}$ & ... \\
			: & : &:& : & : & : &:\\
			$x_k$ & $x_{k0}$ & $x_{k1}$ & $x_{k2}$ & ... & $x_{kk}$ & ... \\
			: & : &:& : & : & : &:\\
			\hline
		\end{tabular}
	\item Sélection de la diagonale (celle-ci est un nombre réel compris
		entre 0 et 1)
		\[ d=0.x_{00}x_{11}...x_{kk}... \]
	\item Modification de cet élément $d$ pour obtenir
		\[ d'=0.x_{00}'x_{11}'...x_{kk}'... \]
		Où $x_{ii}'=5$ si $x_{ii}'\neq 5$ \\
		$x_{ii}'=6$ si $x_{ii}'= 5$ \\
		On a toujours que cet élément $d'$ est compris entre 0 et 1
	\item Contradiction, car $d'$ est dans l'énumération, car $E$ est
		énumérable (par supposition). Il existe donc $x_p=d'$,
		\[ d'=x_p=0.x_{p0}x_{p1}...x_{pp}... \]
		\[ d'=0.x_{00}'x_{11}'...x_{pp}'... \]
		La contradiction vient du fait qu'on a choisi $x_{pp}' \neq
	       	x_{pp}$. Donc $d' \neq x_p$ ce qui implique que $d'$ n'est pas
		dans l'énumération.
	\item Conclusion : $E$ n'est pas énumérable

\end{enumerate}
% subsection cantor (end)

\subsection{Au delà du non énumérable}
\label{subsec:au_dela_du_non_enumerable}
On a d'abord parlé d'ensembles finis, ensuite nous avons introduit le concept d'infini 
énumérable qui est plus grand que tous les ensembles finis. Mais il existe plus grand 
que l'infini énumérable, comme on vient de le démontrer il existe des ensembles infinis 
non énumérables. Est-ce que on s'arête là ou peut-on encore aller plus loin?

Les mathématiciens ont décidés de continuer à chercher plus grand et pour ça, ils ont 
définit l'ensemble des ensembles non énumérables. Et ils ont démontré qu'il était plus 
grand que les ensembles non énumérables. On peut prendre pour exemple l'ensembles des 
ensembles contenant un nombre non énumérables de réels. On peut noter l'ensemble de ces 
ensembles $2^{\R}$. Il peut également être vu comme l'ensemble des fonctions qui 
prennent un entier en entrée et qui donnent un entier en sortie.

Et ce n'est pas tout, les mathématiciens ont définit la notation $2^{\R}$ pour 
cet ensemble mais pourquoi pas prendre un ensemble qui serait l'union de tous ses 
sous-ensembles? On le note $2^{2^{\R}}$ (On peut voir cet ensemble comme toutes 
les ``fonctions'' qui prennent une fonction en entrée et qui donnent une fonction en 
sortie). Et on peut monter aussi haut que l'on veut comme ceci.

On peut ainsi schématiser la taille des ensembles comme ceci:
$$\Phi < \{1,2,3\} < \N < \R < 2^{\R} < 2^{2^{\R}} < 2^{2^{2^{\R}}} \dots$$

Est-ce qu'il existe encore plus grand? Oui, effectivement, l'union de tous ces ensembles 
est plus grand que chacun de ces ensembles.


\section{Conclusion}

Les ensembles énumérables sont importants pour la suite du cours et aussi, car en
informatique on ne considère que les ensembles énumérables.
Dans le cours, on va souvent devoir montrer qu'un ensemble est énumérable/non énumérable.
Généralement on va utiliser une des techniques suivantes:
\begin{itemize}
	\item montrer qu'il y a une bijection avec $\N$ ou $\R$
	\item montrer que l'ensemble est fini
	\item utiliser la diagonalisation (cf. Cantor)
	\item écrire un programme qui énumère l'ensemble
\end{itemize}

% subsection conclusion (end)
% section concepts (end)

